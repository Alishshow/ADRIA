\documentclass[a4paper, 10pt, ]{article}

\usepackage[slovak]{babel}

\usepackage[utf8]{inputenc}
\usepackage[T1]{fontenc}

\usepackage[left=4cm,
			right=4cm,
			top=2.1cm,
			bottom=2.6cm,
			footskip=7.5mm,
			twoside,
			marginparwidth=3.5cm,
			%showframe,
			]{geometry}

\usepackage{graphicx}
\usepackage{xcolor}

% ------------------------------

\usepackage{lmodern}

\usepackage[tt={oldstyle=false,proportional=true,monowidth}]{cfr-lm}

% ------------------------------

\usepackage{amsmath}
\usepackage{amssymb}
\usepackage{amsthm}

\usepackage{booktabs}
\usepackage{multirow}
\usepackage{array}
\usepackage{dcolumn}

\usepackage{natbib}

\usepackage[singlelinecheck=true]{subfig}


% ------------------------------


\usepackage{sectsty}
\allsectionsfont{\sffamily}


\usepackage{titlesec}
\titleformat{\paragraph}[hang]{\sffamily  \bfseries}{}{0pt}{}
\titlespacing*{\paragraph}{0mm}{3mm}{1mm}


\usepackage{fancyhdr}
\fancypagestyle{plain}{%
\fancyhf{} % clear all header and footer fields
\fancyfoot[C]{\sffamily {\bfseries \thepage}\ | {\scriptsize\oznacenieCasti}}
\renewcommand{\headrulewidth}{0pt}
\renewcommand{\footrulewidth}{0pt}}
\pagestyle{plain}


% ------------------------------


\makeatletter

	\def\@seccntformat#1{\protect\makebox[0pt][r]{\csname the#1\endcsname\hspace{5mm}}}

	\def\cleardoublepage{\clearpage\if@twoside \ifodd\c@page\else
	\hbox{}
	\vspace*{\fill}
	\begin{center}
	\phantom{}
	\end{center}
	\vspace{\fill}
	\thispagestyle{empty}
	\newpage
	\if@twocolumn\hbox{}\newpage\fi\fi\fi}

	\newcommand\figcaption{\def\@captype{figure}\caption}
	\newcommand\tabcaption{\def\@captype{table}\caption}

\makeatother


% ------------------------------


\def\naT{\mathsf{T}}

\hyphenpenalty=6000
\tolerance=6000


% ------------------------------


\usepackage[pdfauthor={MT},
			pdftitle={AR},
			pdfsubject={},
			pdfkeywords={},
			linkbordercolor = white,
			%citebordercolor = white,
			breaklinks,
			]{hyperref}


\def\oznacenieCasti{AR04 - LS2020}





\begin{document}





\fontsize{12pt}{22pt}\selectfont

\centerline{\textsf{Adaptívne riadenie} \hfill \textsf{\oznacenieCasti}}

\fontsize{18pt}{22pt}\selectfont





\begin{flushleft}
	\textbf{\textsf{O samonastavujúcom sa regulátore,\newline časť druhá}}
\end{flushleft}





\normalsize

\bigskip

\tableofcontents

\bigskip

\vspace{18pt}




\noindent
Zákon riadenia (štruktúra riadenia) samonastavujúceho sa regulátora v~uvažovanom konkrétnom príklade je v tvare
\begin{subequations} \label{zakonRiadenia}
	\begin{align}
		R(z^{-1})u(k) & = T(z^{-1})r(k) - S(z^{-1})y(k) \\
		u(k) & = \frac{T(z^{-1})}{R(z^{-1})}r(k) - \frac{S(z^{-1})}{R(z^{-1})}y(k)
	\end{align}
\end{subequations}
kde $R$, $S$ a $T$ sú polynómy v tvare
\begin{subequations}
	\begin{align}
		R(z^{-1}) & = 1 + r_1z^{-1} + r_2z^{-2} + \ldots +  r_{n_r}z^{-n_r}\\
		S(z^{-1}) & = s_0 + s_1z^{-1} + s_2z^{-2} + \ldots +  s_{n_s}z^{-n_s} \\
		T(z^{-1}) & = t_0 + t_1z^{-1} + t_2z^{-2} + \ldots +  t_{n_t}z^{-n_t}
	\end{align}
\end{subequations}
Ako už bolo uvedené, koeficienty polynómov sú parametrami regulátora. Počet parametrov regulátora závisí od stupňa jednotlivých polynómov. Pre uvažovaný konkrétny príklad sú stupne polynómov nasledovné: $n_r = 1$, $n_s = 1$ a $n_t = 0$. Potom počet parametrov regulátora je $n_r + n_s + 1 + n_t + 1$. Teda $1 + 1 + 1 + 0 + 1 = 4$. Zákon riadenia je možné zapísať aj v tvare diferenčnej rovnice:
\begin{equation}
	u(k) =  - r_1 u(k-1) - s_0 y(k) - s_1 y(k-1) + t_0 r(k)
\end{equation}



\section{Metóda rozmiestňovania pólov}

Uvažujme zákon riadenia v tvare \eqref{zakonRiadenia}, ktorého parametre budeme počítať pomocou metódy rozmiestňovania pólov. Najprv odvodíme rovnicu uzavretého obvodu (URO).




\subsection{Rovnica URO}


Model riadeného systému je
\begin{equation} \label{modelRiadSys}
	A(z^{-1})y(k)
	=
		B(z^{-1})u(k)
\end{equation}
Dosadením \eqref{zakonRiadenia} do \eqref{modelRiadSys} máme
\begin{equation}
	A(z^{-1})y(k) = B(z^{-1}) \left( \frac{T(z^{-1})}{R(z^{-1})}r(k) - \frac{S(z^{-1})}{R(z^{-1})}y(k) \right)
\end{equation}
Úpravou
\begin{subequations}
	\begin{align}
		A y(k) &= \frac{BT}{R} r(k) - \frac{BS}{R} y(k) \\
		RA y(k) &= BT r(k) - BS y(k)\\
		\left( RA + BS \right) y(k) &= BT r(k) \\
		y(k) &= \frac{BT}{\left( AR + BS \right)} r(k) \\
		\frac{y(k)}{r(k)} &= \frac{BT}{\left( AR + BS \right)}
	\end{align}
\end{subequations}
Charakteristický polynóm uzavretého regulačného obvodu je:
\begin{equation}
	A(z^{-1})R(z^{-1}) + B(z^{-1})S(z^{-1})
\end{equation}


Nech želaný polynóm je
\begin{equation}
	P(z^{-1})  = 1 + p_1z^{-1} + p_2z^{-2} + \ldots +  p_{n_p}z^{-n_p}
\end{equation}
potom diofantická rovnica, z ktorej sa vypočítajú koeficienty polynómov $R$ a $S$ je
\begin{equation} \label{prvaDiofantRovn}
	A(z^{-1})R(z^{-1}) + B(z^{-1})S(z^{-1})  = 	P(z^{-1})
\end{equation}




V tomto prípade máme
\begin{subequations}
	\begin{align}
		A & = 1 + a_1z^{-1} + a_2z^{-2} \\
		B & = b_1z^{-1} + b_2z^{-2} \\
		R & = 1 + r_1z^{-1} \\
		S & = s_0 + s_1z^{-1}
	\end{align}
\end{subequations}
a nech želaný polynóm pre tento prípad je
\begin{equation}
	\begin{split}
		P & = 1 + p_1z^{-1} + p_2z^{-2}
	\end{split}
\end{equation}
Diofantická rovnica pre tento prípad
\begin{equation}
	\begin{split}
		& \left( 1 + a_1z^{-1} + a_2z^{-2} \right) \left( 1 + r_1z^{-1} \right) + \left( b_1z^{-1} + b_2z^{-2} \right) \left( s_0 + s_1z^{-1} \right) \\
		&= 1 + p_1z^{-1} + p_2z^{-2}
	\end{split}
\end{equation}
Roznásobením
\begin{equation}
	\begin{split}
		& 1+r_1z^{-1}+a_1z^{-1} + a_1r_1z^{-2}+a_2z^{-2}+a_2r_1z^{-3} \\
		&+ b_1s_0z^{-1}+b_1s_1z^{-2} + b_2s_0z^{-2}+b_2s_1z^{-3} \\
		&= 1 + p_1z^{-1} + p_2z^{-2}
	\end{split}
\end{equation}
Na ľavej strane ponecháme členy, v ktorých sa nachádzajú neznáme koeficienty polynómov zo zákona adaptácie a ostatné členy presunieme na pravú stranu
\begin{equation}
	\begin{split}
		& r_1z^{-1} + a_1r_1z^{-2}+a_2r_1z^{-3} + b_1s_0z^{-1}+b_1s_1z^{-2} + b_2s_0z^{-2}+b_2s_1z^{-3} \\
		&= 1 + p_1z^{-1} + p_2z^{-2} - 1 -a_1z^{-1} -a_2z^{-2}
	\end{split}
\end{equation}
Po úprave
\begin{equation}
	\begin{split}
		& \left({ r_1 + b_1s_0 }\right) z^{-1} + \left({ a_1r_1 +b_1s_1 + b_2s_0 }\right) z^{-2} + \left({ a_2r_1 +b_2s_1 }\right) z^{-3} \\
		&= \left({ p_1 -a_1 }\right) z^{-1} + \left({p_2 -  a_2 }\right) z^{-2}
	\end{split}
\end{equation}
Porovnaním koeficientov pri rovnakých mocninách získame rovnice
\begin{subequations}
	\begin{align}
		 r_1 + b_1 s_0  &= p_1 - a_1 \\
		 a_1 r_1 + b_2 s_0 + b_1 s_1  &= p_2 -  a_2 \\
		 a_2 r_1  + b_2 s_1  &= 0
	\end{align}
\end{subequations}
V maticovom zápise:
\begin{equation}
	\begin{bmatrix} 1 & b_1 & 0 \\ a_1 & b_2 & b_1 \\ a_2 &   0 & b_2 \end{bmatrix}
	\begin{bmatrix} r_1 \\ s_0 \\ s_1  \end{bmatrix}
	=
	\begin{bmatrix} p_1 - a_1 \\ p_2 - a_2 \\ 0 \end{bmatrix}
\end{equation}






Maticový zápis vyplývajúci z diofantickej rovnice v prípade, keď stupne polynómov $R$, $S$ a $P$ sú všeobecné, je v tvare
\begin{equation} \label{VelkaDiofanRov}
	\begin{bmatrix}
		   1 &      0 &      0 & \cdots &       0 &     b_1 &      0 &      0 & \cdots &       0 \\
	 	 a_1 &      1 &      0 & \cdots &       0 &     b_2 &    b_1 &      0 & \cdots &       0 \\
	 	 a_2 &    a_1 &      1 & \cdots &       0 &     b_3 &    b_2 &    b_1 & \cdots &       0 \\
      \vdots & \vdots & \vdots & \ddots &  \vdots &  \vdots & \vdots & \vdots & \ddots &  \vdots \\
     a_{n_a} & \vdots & \vdots & \cdots &       1 & b_{n_b} & \vdots & \vdots & \cdots &     b_1 \\
           0 & \ddots & \vdots & \cdots &     a_1 &       0 & \ddots & \vdots & \cdots &  \vdots \\
      \vdots & \vdots & \ddots & \cdots &  \vdots &  \vdots & \vdots & \ddots & \cdots &  \vdots \\
      \vdots & \vdots & \vdots & \ddots &  \vdots &  \vdots & \vdots & \vdots & \ddots &  \vdots \\
           0 &      0 &      0 & \cdots & a_{n_a} &       0 &      0 &      0 & \cdots & b_{n_b}
	\end{bmatrix}
	\begin{bmatrix} r_1 \\ r_2 \\ r_2 \\ \vdots \\ r_{n_r} \\ s_0 \\ s_1 \\ s_2 \\ \vdots \\ s_{n_s} \end{bmatrix}
	=
	\begin{bmatrix} p_1 - a_1 \\ p_2 - a_2 \\ \vdots \\ 0 \\ \vdots \\ 0 \end{bmatrix}
\end{equation}


Takáto sústava rovníc bude mať riešenie ak
\begin{subequations}
	\begin{align}
		 n_r &= n_b - 1 \\
		 n_s &= n_a - 1
	\end{align}
\end{subequations}







\subsection{Polynóm $T$}

\bigskip

Zatiaľ sme vypočítali koeficienty polynómov $R$ a $S$. Otázkou ostáva, ako určiť polynóm $T$. Je potrebné určiť jeho stupeň a vypočítať koeficienty. V úvode sme \uv{zvolili}, že stupeň polynómu $T$ je $n_t = 0$. Teda jediným koeficientom bude $t_0$. Ukážme teraz, že jednou z možností, ako určiť polynóm $T$, je žiadať nulovú regulačnú odchýlku v ustálenom stave. Dostaneme tak polynóm $T$ práve nultého stupňa a aj výpočet koeficientu $t_0$.

Keďže charakteristická rovnica URO je rovnaká ako želaný polynóm $P$, je možné písať rovnicu uzavretého obvodu v tvare
\begin{equation}
	y(k) = \frac{BT}{P} r(k)
\end{equation}
Aby platilo
\begin{equation}
	y(\infty) = r(\infty)
\end{equation}
musí byť
\begin{subequations}
	\begin{align}
		BT &= P \\
		T &= \frac{P}{B}
	\end{align}
\end{subequations}
A keďže \uv{donekonečna} je v diskrétnej doméne \uv{dojednotky}, teda $z = 1$, potom
\begin{equation}
	T = \frac{P(1)}{B(1)}
\end{equation}
čo v tomto prípade znamená
\begin{equation}
	T(z^{-1}) = \frac{1 + p_1 + p_2}{b_1 + b_2} = t_0
\end{equation}





\subsubsection{Alternatívy spôsob určenia polynómu $T$}

\paragraph{Alternatíva 1}

Ďalší spôsob ako určiť koeficienty polynómu $T$ je nasledovný. Nájdeme obraz referenčného signálu tak, že jeho časovú formu pretransformujeme pomocou Z-transformácie. Totiž v mnohých prípadoch je možné dopredu určiť časový priebeh referenčného signálu a~navyše sa referenčný signál skladá zo skokov, rámp a~podobných, pomocou Z-transformácie, ľahko transformovateľných signálov. Obraz referenčného signálu je
\begin{equation}
	\left\{ r(t) \right\}_q = \frac{F \left( q^{-1} \right)}{G \left( q^{-1} \right)}
\end{equation}
Tento obraz použijeme vo vzťahu pre regulačnú odchýlku:
\begin{equation}
	e = r - y = r - \frac{BT}{P}r = \frac{F}{G} - \frac{BT}{P}\frac{F}{G} = \frac{F(P - BT)}{GP} = \frac{FN}{P}
\end{equation}
kde sme označili
\begin{equation}
	\frac{P - BT}{G} = N
\end{equation}
Z tohto označenia môžeme písať diofantickú rovnicu, ktorá doplní \eqref{prvaDiofantRovn}, a vznikne tak sústava.
\begin{equation}
	GN + BT = P
\end{equation}
Z tejto rovnice sa dá určiť aj polynóm vyššieho ako nultého stupňa.



\paragraph{Alternatíva 2}

Ďalší spôsob ako určiť koeficienty polynómu $T$ je takýto: ak bude polynóm $T$~obrátenou hodnotou polynómu $B$, teda $T = 1/B$, zaistí sa tak nie len nulová regulačná odchýlka v ustálenom stave, ale aj to, že polynóm $B$ nebude mať žiadny vplyv na dynamiku URO. Dynamika URO bude daná len želaným charakteristickým polynómom $P$, takto:
\begin{equation}
	y(t) = \frac{1}{P}r(t)
\end{equation}
Zákon riadenia potom môžeme uvažovať v tvare
\begin{equation}
	Ru(t) = \frac{1}{B}r(t) - Sy(t) \quad \Rightarrow \quad r = BRu + BSy
\end{equation}
Ale ak $B = q^{-D}\tilde{B}$, tak aby sme mohli napísať predchádzajúcu rovnicu musíme dať $q^{-D}$ na druhú stranu k $r$. Teda:
\begin{equation}
	 rq^{D} = \tilde{B}Ru + \tilde{B}Sy
\end{equation}
Ak potom vyjadríme akčný zásah, je zrejmé, že bude závisieť od budúcich hodnôt referenčného signálu $u(t) = r(t + D)-\ldots$ Toto však nemusí byť prekážkou, pretože v mnohých prípadoch je referenčný signál dopredu známy.




\subsection{Súhrn pre tento prípad}

Zhrňme výpočty potrebné pre určenie parametrov regulátora pre tento prípad.

Hodnoty parametrov modelu $a_1$, $a_2$, $b_1$ a $b_2$ sú po identifikácii (v danej perióde vzorkovania) známe a prirodzene sú známe aj hodnoty koeficientov želaného polynómu $P$. Hodnoty parametrov regulátora získame riešením
\begin{equation}
	\begin{bmatrix} 1 & b_1 & 0 \\ a_1 & b_2 & b_1 \\ a_2 &   0 & b_2 \end{bmatrix}
	\begin{bmatrix} r_1 \\ s_0 \\ s_1  \end{bmatrix}
	=
	\begin{bmatrix} p_1 - a_1 \\ p_2 - a_2 \\ 0 \end{bmatrix}
\end{equation}
a
\begin{equation}
		t_0 = \frac{1 + p_1 + p_2}{b_1 + b_2}
\end{equation}
















\section{Rýchlostný algoritmus metódy rozmiestňovania pólov}

Keďže ide o rýchlostný algoritmus, je potrebné vyjadriť prírastok akčnej veličiny:
\begin{align}
	 \Delta u(t) &= u(t) - u(t-1) = (1 - q^{-1})u(t) \\
	 u(t) &= \frac{\Delta u(t)}{(1 - q^{-1})}
\end{align}
Riadený systém a jeho ARX model (s nulovým šumom):
\begin{equation}
	Ay(t) = Bu(t)
\end{equation}
do ktorého dosadíme $u(t)$ a upravíme\ldots
\begin{subequations}
	\begin{align}
		 Ay(t) &= B\frac{\Delta u(t)}{(1 - q^{-1})} \\
		 (1 - q^{-1})Ay(t) &= B\Delta u(t)
	\end{align}
\end{subequations}
Zákon riadenia uvažujeme v tvare:
\begin{equation}
	\Delta u(t) = \frac{S}{R}(r(t) - y(t))
\end{equation}
Rovnica URO potom bude mať tvar
\begin{subequations}
	\begin{align}
		 (1 - q^{-1})Ay(t) &= B\Delta u(t) \\
		 (1 - q^{-1})Ay(t) & = B \frac{S}{R}(r(t) - y(t)) \\
		 (1 - q^{-1})ARy(t) &= BS(r(t) - y(t)) \\
		 (1 - q^{-1})ARy(t) &= BSr(t) - BSy(t) \\
		 ((1-q^{-1})AR + BS) y(t) &= BSr(t) \\
		 y(t) &= \frac{BS}{(1-q^{-1})AR + BS}r(t)
	\end{align}
\end{subequations}
Takže charakteristický polynóm je
\begin{equation}
	P = \left( 1 - q^{-1} \right) AR + BS
\end{equation}
Ale veď potom
\begin{equation}
	BS = P - \left( 1 - q^{-1} \right)AR
\end{equation}
a tak rovnica URO:
\begin{subequations}
	\begin{align}
		y(t) &= \frac{P - (1-q^{-1})AR}{P}r(t) \\
		y(t) &=\left({\frac{P}{P} -  \frac{(1-q^{-1})AR}{P}}\right)r(t) \\
		y(t) &= r(t) -  \frac{(1-q^{-1})AR}{P}r(t)
	\end{align}
\end{subequations}
Ak sa teraz opýtame aká bude regulačná odchýlka v ustálenom stave t.j. $y(\infty)$, $r(\infty)$ a čo je najdôležitejšie $q = 1$ potom:
\begin{equation}
	\begin{split}
		& y(\infty) =r(\infty) -  \frac{(1-1)AR}{P}r(\infty)
		\\& y(\infty) =r(\infty)
	\end{split}
\end{equation}
Teda regulačná odchýlka v ustálenom stave bude nulová (pretože sa zhodujú hodnoty referenčného signálu a~výstupnej veličiny).











\section{Cvičenie tretie}
\label{cvictretie}


\begin{enumerate}[leftmargin=0pt, labelsep=4mm, itemsep=0pt]

	\item Zrealizujte (v simulačnom prostredí) samonastavujúci sa regulátor tak ako to predpokladá uvažovaný konkrétny príklad.

	Nech želaným charakteristickým polynómom (pre uvažovaný konkrétny príklad) je $P(z^{-1}) = 1 + p_1z^{-1} + p_2z^{-2}$ pričom $p_1 = -1,6$ a~$p_2 = 0,64$, teda dvojnásobný koreň $z_{1,2} = 0,8$.

	Alternatívne, nech v želanom charakteristickom je $p_1 = -0,8$ a~$p_2 = 0,16$, teda dvojnásobný koreň $z_{1,2} = 0,4$.

	Pozn: pre uvažovaný príklad sa odporúča perióda vzorkovania $T_{vz} = 0,1$ [s].

\end{enumerate}









\section{Otázky a úlohy}
% \addcontentsline{toc}{section}{Otázky a úlohy}



\begin{enumerate}[leftmargin=0pt, labelsep=4mm, itemsep=0pt]

	\item Modelom riadeného systému je ARX model. Zákon riadenia má tvar $R(z^{-1}) u(k) =  T(z^{-1}) r(k) -  S(z^{-1}) y(k)$. Nájdite charakteristický polynóm URO.

	\item Modelom riadeného systému je ARX model. Zákon riadenia má tvar $ u(k) =  \Delta u(k) / (1 - z^{-1})$, kde
	\begin{equation*}
		\Delta u(k) = \frac{S(z^{-1})}{R(z^{-1})} (r(k) - y(k))
	\end{equation*}
		Nájdite charakteristický polynóm URO.

	\item Stručne vysvetlite výpočet parametrov regulátora metódou pole-placement.

	\item Vysvetlite podstatu metód návrhu polynómu $T(z^{-1})$ pri STR.

\end{enumerate}







\end{document}
