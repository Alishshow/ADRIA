\documentclass[a4paper, 10pt, ]{article}

\usepackage[slovak]{babel}

\usepackage[utf8]{inputenc}
\usepackage[T1]{fontenc}

\usepackage[left=4cm,
			right=4cm,
			top=2.1cm,
			bottom=2.6cm,
			footskip=7.5mm,
			twoside,
			marginparwidth=3.5cm,
			%showframe,
			]{geometry}

\usepackage{graphicx}
\usepackage{xcolor}

% ------------------------------

\usepackage{lmodern}

\usepackage[tt={oldstyle=false,proportional=true,monowidth}]{cfr-lm}

% ------------------------------

\usepackage{amsmath}
\usepackage{amssymb}
\usepackage{amsthm}

\usepackage{booktabs}
\usepackage{multirow}
\usepackage{array}
\usepackage{dcolumn}

\usepackage{natbib}

\usepackage[singlelinecheck=true]{subfig}


% ------------------------------


\usepackage{sectsty}
\allsectionsfont{\sffamily}


\usepackage{titlesec}
\titleformat{\paragraph}[hang]{\sffamily  \bfseries}{}{0pt}{}
\titlespacing*{\paragraph}{0mm}{3mm}{1mm}


\usepackage{fancyhdr}
\fancypagestyle{plain}{%
\fancyhf{} % clear all header and footer fields
\fancyfoot[C]{\sffamily {\bfseries \thepage}\ | {\scriptsize\oznacenieCasti}}
\renewcommand{\headrulewidth}{0pt}
\renewcommand{\footrulewidth}{0pt}}
\pagestyle{plain}


% ------------------------------


\makeatletter

	\def\@seccntformat#1{\protect\makebox[0pt][r]{\csname the#1\endcsname\hspace{5mm}}}

	\def\cleardoublepage{\clearpage\if@twoside \ifodd\c@page\else
	\hbox{}
	\vspace*{\fill}
	\begin{center}
	\phantom{}
	\end{center}
	\vspace{\fill}
	\thispagestyle{empty}
	\newpage
	\if@twocolumn\hbox{}\newpage\fi\fi\fi}

	\newcommand\figcaption{\def\@captype{figure}\caption}
	\newcommand\tabcaption{\def\@captype{table}\caption}

\makeatother


% ------------------------------


\def\naT{\mathsf{T}}

\hyphenpenalty=6000
\tolerance=6000


% ------------------------------


\usepackage[pdfauthor={MT},
			pdftitle={AR},
			pdfsubject={},
			pdfkeywords={},
			linkbordercolor = white,
			%citebordercolor = white,
			breaklinks,
			]{hyperref}


\def\oznacenieCasti{AR00 - LS2019}



\usepackage{pdflscape}

\usepackage{longtable}



\begin{document}





\fontsize{12pt}{22pt}\selectfont

\centerline{\textsf{Adaptívne riadenie} \hfill \textsf{\oznacenieCasti}}

\fontsize{18pt}{22pt}\selectfont





\begin{flushleft}
	\textbf{\textsf{Organizácia predmetu}}
\end{flushleft}





\normalsize

\bigskip



\noindent
Adaptívne riadenie (LS, ak.r. 2018/2019)

\medskip

\noindent
Cieľ predmetu: \newline
Študenti po absolvovaní predmetu získajú vedomosti o~najvýznamnejších metódach a~algoritmoch používaných v~oblasti adaptívneho riadenia procesov. Absolventi predmetu získajú vedomosti týkajúce sa odvodenia a~analýzy vlastností vybraných algoritmov priameho adaptívneho riadenia a nepriameho adaptívneho riadenia. Získajú poznatky o~základných princípoch vybraných heuristických adaptívnych regulátorov, komerčných adaptívnych regulátorov, a~princípoch využitia adaptácie pri fuzzy riadiacich systémoch.

\medskip

\noindent
Zodpovedný za predmet: prof. Ing. Ján Murgaš, PhD.

\medskip

\noindent
Predmet patrí medzi povinné predmety a študent po absolvovaní získa 7~kreditov. Týždenný rozsah predmetu: prednášky: 2 h, cvičenia: 2 h

\bigskip

\noindent
\textbf{\textsf{Predmet zabezpečujú:}}

\medskip

\noindent
Ing. Marián Tárník, PhD. (prednášky, cvičenia)

\bigskip

\noindent
\textbf{\textsf{Podmienky absolvovania predmetu:}}

\noindent
\begin{enumerate}

	\item Aktívna účasť na vyučovacom procese.

	\item Počas semestra je možné získať max. 60 bodov, pričom pre splnenie podmienok pre vykonanie skúšky je potrebných 33,6 bodu.

	\item Je potrebná účasť na záverečnej skúške, je možné získať max. 40 bodov.

\end{enumerate}

\bigskip

\noindent
\textbf{\textsf{Priebežné hodnotenie študentov dennej prezenčnej formy štúdia počas semestra:}}

\noindent
\begin{itemize}
	\item Priebežná práca na cvičeniach:  19 bodov
	\item Krátka písomka na cvičeniach:  6 bodov
	\item Písomka v čase 7. prednášky:  20 bodov
	\item Vypracovanie zadania (referát):  15 bodov
\end{itemize}


\bigskip

\noindent
\textbf{\textsf{Priebežné hodnotenie študentov dennej dištančnej formy štúdia počas semestra:}}

\noindent
\begin{itemize}
	\item Vypracovanie 4 zadaní (referát):  4 $\times$ 10 bodov
	\item Písomka:  20 bodov
\end{itemize}






\nocite{*}
\bibliography{misc/Bib_KurzAR}{}
\bibliographystyle{plain}










\begin{landscape}




    \phantom{}

    \vfill



    \noindent
    \textbf{\textsf{Harmonogram semestra pre študentov dennej prezenčnej formy štúdia}}

    \bigskip


    \noindent
    \begin{tabular*}{\linewidth}{   >{\raggedright}p{1.2cm} @{\extracolsep{\fill}}    >{\raggedright}p{10.9cm}        p{11.6cm}<{\raggedright}      }
    	Týždeň             & Prednáška & Cvičenie     \\
    	\toprule
    	1.  &  Úvod, stabilita systémov, adaptívna stabilizácia. & Adaptívna stabilizácia (1b) \\
    	\midrule
    	2.  & Samonastavujúci sa regulátor: rekurzívna metóda najmenších štvorcov. & Samonastavujúci sa regulátor: rekurzívna metóda najmenších štvorcov (2b) \\
    	\midrule
    	3.  & Samonastavujúci sa regulátor (info k cv), riadenie s~referenčným modelom. & Samonastavujúci sa regulátor: metóda rozmiestňovania pólov (2b) \\
    	\midrule
    	4.  & MRAC gradientný. & Krátka písomka (6b), prípadné dokončenie predchádzajúcich úloh \\
    	\midrule
    	5.  & MRAC~gradientný (info k cv.), MRAC stavový.  & MRAC~gradientný (2b) \\
    	\midrule
    	6.  & MRAC stavový (info k cv.), riadenie s referenčným modelom vo všeobecnosti - MRC~problém. & MRAC stavový (3b) \\
    	\midrule
    	7.  & Písomka (prezenčná f.) (20b) & MRC problém, prípadné dokončenie predchádzajúcich úloh \\
    	\midrule
    	8.  & MRAC vstupno-výstupný pre $n^\star = 1$. & MRAC vstupno-výstupný pre $n^\star = 1$ (5b) \\
    	\midrule
    	9.  &  \#ŠVOČ & Poznámky k robustnému adaptívnemu riadeniu (vplyv šumu a~nemodelovanej dynamiky) \\
    	\midrule
    	10. & MRAC vstupno-výstupný pre $n^\star = 2$, príklad.  &  MRAC vstupno-výstupný pre $n^\star = 2$: vypracovanie zadania (referátu) I\\
    	\midrule
    	11. & Rôzne. & MRAC vstupno-výstupný pre $n^\star = 2$: vypracovanie zadania (referátu) II  \\
    	\midrule
    	12. & 1. máj  & AR pre kyvadlo (kyvadlo ako riadený systém) (4b)  \\
    	\midrule
    	13. & Streda v utorok (40b ak sa podarí) & Udeľovanie „zápočtov“ {\scriptsize \emph{(nech už to znamená čokoľvek)}} \\
    	\bottomrule
    \end{tabular*}

    \vfill

    \phantom{}






    \pagebreak











\pagebreak

\phantom{}

\vfill




\noindent
\textbf{\textsf{Harmonogram semestra pre študentov dennej dištančnej formy štúdia}}

\bigskip


\noindent
\begin{tabular*}{\linewidth}{   >{\raggedright}p{1.2cm} @{\extracolsep{\fill}}    >{\raggedright}p{11.5cm}        p{11.0cm}<{\raggedright}      }
	Týždeň            & Prednáška (uvedený je len harmonogram základných tém)  & Konzultácia    \\
	\toprule
	1.   &  &  Adaptívna stabilizácia - zadanie 1. (10b)  \\
	\midrule
	2.  & Samonastavujúci sa regulátor   &  \\
	\midrule
	3.  &  & Samonastavujúci sa regulátor - zadanie 2. (10b)   \\
	\midrule
	4.  & MRAC gradientný   &  \\
	\midrule
	5.  & MRAC stavový  & MRAC~gradientný - zadanie 3. (10b)    \\
	\midrule
	6.  &  &   \\
	\midrule
	8.  & MRAC vstupno-výstupný pre $n^\star = 1$  & Písomka (dištančná f.) (20b)   \\
	\midrule
	8.  &  &   \\
	\midrule
	9. &  & MRAC stavový - zadanie 4. (10b)     \\
	\midrule
	10. & MRAC vstupno-výstupný pre $n^\star = 2$  &   \\
	\midrule
	11. &  & MRAC vstupno-výstupný    \\
	\midrule
	12. &  &    \\
	\bottomrule
\end{tabular*}


\vfill

\phantom{}



\end{landscape}













\end{document}
